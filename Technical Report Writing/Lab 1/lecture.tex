\documentclass[a4paper, 12pt]{article}
\usepackage[margin=1in]{geometry}
\usepackage[]{hyperref}

\usepackage{amsmath}
\usepackage{amscd}
\usepackage{amssymb}
\begin{document}
	\section{Modes}
	mathematical equation
	$ (a+b)^2 = \ a^2+2ab+b^2$
	
	\begin{math}
		(a+b)^2 = a^2 + 2ab + b^2
	\end{math}
	
	we can also write function as 
	\begin{displaymath}
		(a+b)^2 = a^2 + 2ab + b^2
	\end{displaymath}
	or with
	
	\begin{equation}
		a + b = 2
	\end{equation}
	\begin{equation}
		a + b = 2
	\end{equation}
	\begin{equation} \label{eqn:something}
		a + b = 2
	\end{equation}
	
	So for in \autoref{eqn:something} we have an equation and in \ref{eqn:something}
	
	\section{Superscript and Subscript}
	\begin{equation}
		x_i = 10	
	\end{equation}
	\begin{equation}
		x^{2000}_{(i,j)} = 10
	\end{equation}
	\begin{equation}
		1 + 2 = 5 -2 = 6/2 = 1.5 \times 2
	\end{equation}
	\begin{equation}
		sin^2\theta + cos^2\theta = 1
	\end{equation}
	\begin{equation*}
		\sin^2\theta + \cos^2\theta = \log10
		\text{, This is a simple equation}
	\end{equation*}
	We can write $\frac{a}{b} = \frac{c}{d}$
	\begin{equation}
		\frac{a}{\frac{3}{6}} = \frac{c}{d}
	\end{equation}
	\begin{equation}
		\sqrt{\frac{a}{b}} = \frac{1}{\sqrt{2}}
	\end{equation}
	\begin{equation}
		\left[
		\sqrt{\frac{a + 5 \times b}{\frac{2+c}{d}}} = \frac{1}{\sqrt{2}}
		\right]
	\end{equation}
	
	\section{Greek Alphabets}
	\begin{equation}
		\alpha \beta \gamma \delta \Gamma \Delta
	\end{equation}
	
	\section{Calculus}
	\begin{equation}
		\lim_{x \rightarrow \infty} \frac{1}{x} = 0
	\end{equation}
	
	\begin{equation}
		\int_{10}^{200} x dx = 100
	\end{equation}
	\begin{equation}
		\sum_{i=0}^{200} x_i = \prod_{i=0}^{200} (x_i + 1)
	\end{equation}
	
	\section{Caligraphy}
	\begin{equation}
		\mathcal{ABCDEF}, \mathfrak{ABCDEF}, \mathbb{ABCDEF}
	\end{equation}
	
	\section{Operators}
	\begin{equation}
		1 = 1 <5 > 2 \neq 4 \leq 8 \geq 16
	\end{equation}
	
	\section{Sets and Vectors}
	\begin{equation}
		A \cup B \cup C \in \mathbb{R}
	\end{equation}
	\begin{equation}
		\hat{i} \times \hat{j} = \hat{k}
	\end{equation}
	\begin{equation}
		\vec{A} \cdot \vec{B} = \vec{C}
	\end{equation}
	
	\begin{multline}
		1 + 2 + 3 + 3 + 5 + 1 + 2 + 3 + \\
		3 + 5 + 1 + 2 + 3 + 3 + 5 + 1 + \\
		2 + 3 + 3 + 5 + 1 + 2 + 3 + 3 + \\
		5 + 1 + 2 + 3 + 3 + 5 + 1 + 2 + \\
		3 + 3 + 5 + 1 + 2 + 3 + 3 + 5 + ... = \infty
	\end{multline}
	\begin{equation}
		\begin{split}
			\frac{a}{b} &= \frac{5}{10} \\
						&= \frac{1}{2} \\
						&= 0.5
		\end{split}
	\end{equation}
	
	\begin{gather}
		a + b + c = 5 \\
		a = 10 \\
		- 2b + 2c = -5
	\end{gather}
	
	\section{Matrix}
	\begin{equation*}
		I = \begin{matrix}
			1 & 0 & 0 \\
			0 & 1 & 0 \\
			0 & 0 & 1
		\end{matrix}
	\end{equation*}
	\begin{equation*}
		I = \begin{bmatrix}
			1 & 0 & 0 \\
			0 & 1 & 0 \\
			0 & 0 & 1
		\end{bmatrix}
	\end{equation*}
	\begin{equation*}
		I = \begin{Bmatrix}
			1 & 0 & 0 \\
			0 & 1 & 0 \\
			0 & 0 & 1
		\end{Bmatrix}
	\end{equation*}
	\begin{equation*}
		I = \begin{pmatrix}
			1 & 0 & 0 \\
			0 & 1 & 0 \\
			0 & 0 & 1
		\end{pmatrix}
	\end{equation*}
\end{document}